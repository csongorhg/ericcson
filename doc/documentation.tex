%
% Encoding: utf-8
% Project:	Implementation of the Lemke-Howson algorithm for finding MNE
% Author:	Petr Zemek <s3rvac@gmail.com>, 2009
%
% Project documentation.
%
\documentclass[a4paper,10pt]{article}

% Packages
\usepackage[utf8]{inputenc}
\usepackage{url}

% View
\setlength{\hoffset}{-1.5cm}
\setlength{\voffset}{-1.5cm}
\setlength{\textheight}{23.0cm}
\setlength{\textwidth}{15.2cm}

% Paragraph formatting - no indents
\setlength{\parskip}{1.3ex plus 0.2ex minus 0.2ex}
\setlength{\parindent}{0pt}

% Text
\begin{document}

% Title
\begin{center}
	\begin{Large}\textbf{Implementation of the Lemke-Howson
	algorithm}\end{Large} \\

	\vspace{0.4cm}

	Petr Zemek, \texttt{s3rvac@gmail.com}

	\vspace{0.4cm}

	\today
\end{center}

\section{Introduction}

This documentation describes my implementation of the Lemke-Howson algorithm
\cite{Lemke64} for finding mixed Nash equilibria in two-person games. For the
sake of clarity, a running example is used to explain the algorithm and the
implementation of it. For a formal description of the algorithm, please refer
to \cite{Lemke64} or \cite{Pritchard08}.

The implementation itself is described in section \ref{sec:Implementation}. The
way how the implementation was tested is outlined in section \ref{sec:Testing}.
Project summarization is given in the conclusion (section
\ref{sec:Conclusion}).

\section{Implementation}
\label{sec:Implementation}

The implementation is written in
\emph{Python}\footnote{\url{http://www.python.org/}}, which is a portable,
object oriented, dynamic scripting language suitable for tasks like this. Used
Python version is 2.5. The following modules from the Python standard library
were used: \texttt{os}, \texttt{pkgutil}, \texttt{re}, \texttt{sys},
\texttt{tempfile} and \texttt{unittest}.

To illustrate the algorithm and corresponding parts of its implementation, the
following game will be used as a running example (player A has strategies 1 and
2 and player B has strategies 3, 4 and 5):

\begin{center}
	\begin{tabular}{c||c|c|c}
		A/B & 3 & 4 & 5 \\
		\hline
		\hline
		1 & 0,1 & 2,-1 & 3,0 \\
		\hline
		2 & 3,-1 & -2,1 & 2,-2 \\
	\end{tabular}
\end{center}

\subsection{Custom supportive modules}

Because I did not find any suitable module in the standard library for working
with matrices and rational numbers (I did not want to use en external numeric
library), I implemented the two following supportive modules from scratch.

\subsubsection{Matrices}

Module \texttt{matrix} provides a representation of two-dimensional matrices
(implemented in a two dimensional array using lists) and a function for parsing
matrices from text. Only basic needed operations over matrices were implemented
(getting and setting values on given index, matrix conversion to its textual
representation).

\subsubsection{Rational numbers}

For the sake of precision of the Lemke-Howson algorithm implementation and
testing (float numbers are hard to test on equality), I implemented my own
representation of rational numbers and basic operations over them. Module
\texttt{rational} provides basic operations over rational numbers, such as
addition, multiplication, negation, division, absolute value, reciprocal value
and conversion from and to textual representation.

\subsection{Input}

The game described in the beginning of section \ref{sec:Implementation} is
passed to the program in the following form (there must be a new line character
at the end of each line):

\begin{verbatim}
0 2 3
3 -2 2

1 -1 0
-1 1 -2
\end{verbatim}

The previous text is parsed (using my modules \texttt{io} and \texttt{matrix})
into the two following matrices:

\begin{minipage}[b]{0.5\linewidth}
	$$A = \left(
	\begin{array}{ccc}
	0 & 2 & 3 \\
	3 & -2 & 2
	\end{array}
	\right)$$
\end{minipage}
\begin{minipage}[b]{0.5\linewidth}
	$$B = \left(
	\begin{array}{ccc}
	1 & -1 & 0 \\
	-1 & 1 & -2
	\end{array}
	\right)$$
\end{minipage}

These two matrices are the input for the Lemke-Howson algorithm.

\subsection{Lemke-Howson algorithm}

The implementation of the Lemke-Howsown algorithm is in the module \texttt{lh}.

\subsubsection{Initialization}

Aside from some limitations of the algorithm (see section
\ref{sec:Limitations}), the used method only works when there are no negative
elements in both matrices and there is no column or row with all zeros
\cite{Pritchard08}. To assure this, a proper constant is added to both matrices
(this does not affect the game and the resulting game is equivalent to the
original game \cite{Pritchard08}). In our example, the lowest value is $-2$, so
we add $3$ to each matrix cell\footnote{In this example, adding $2$ would be
enough, but it might not suffice in all cases, so a greater number is added.}.

\begin{minipage}[b]{0.5\linewidth}
	$$A^{'} = \left(
	\begin{array}{c c c}
	3 & 5 & 6 \\
	6 & 1 & 5
	\end{array}
	\right)$$
\end{minipage}
\begin{minipage}[b]{0.5\linewidth}
	$$B^{'} = \left(
	\begin{array}{c c c}
	4 & 2 & 3 \\
	2 & 4 & 1
	\end{array}
	\right)$$
\end{minipage}

My implementation of the Lemke-Howson algorithm is based on the \emph{Tableaux
method} \cite{Pritchard08}. The two input matrices are converted into a
\emph{tableaux} (a system of linear equations containing data from both
input matrices and additional variables needed in the computation).

From the algebraic point of view, the tableaux corresponding to our game
looks like this (with modified matrices $A^{'}$ and $B^{'}$ to assure the
assumptions described above):

$$\begin{array}{r r r r r r r r}
s_{1} = 1 &         &         & -3x_{3} & -5x_{4} & -6x_{5} \\
s_{2} = 1 &         &         & -6x_{3} & -x_{4}  & -5x_{5} \\
s_{3} = 1 & -4x_{1} & -2x_{2} &         &         &         \\
s_{4} = 1 & -2x_{1} & -4x_{2} &         &         &         \\
s_{5} = 1 & -3x_{1} & -x_{2}  &         &         &
\end{array}$$

$s_{1}, \dots, s_{5}$ are called \emph{slack} variables and $x_{1}, \dots,
x_{5}$ are called \emph{actual} variables. As you can see, the system contains
one slack variable, one actual variable and one equation for each strategy in
the game. Coefficients are negated values from original matrices $A^{'}$ and
$B^{'}$. The leftmost column is called a \emph{basis} and variables that are on
the left side of equations are called \emph{in basis}. Variables on the right
side of the equation are called \emph{out of the basis}. At the beginning, all
slack variables are in the basis and all actual variables are out of the basis.
The ones in the first column on the right side are initial basis values (these
will form the game equilibrium at the end of computation).

From the implementation point of view, the tableaux corresponding to our game
looks like this (again, with modified matrices $A^{'}$ and $B^{'}$ to assure
some conditions):

$$\left(\begin{array}{r r r r r r r}
-1 & 1 & 0 & 0 & -3 & -5 & -6 \\
-2 & 1 & 0 & 0 & -6 & -1 & -5 \\
-3 & 1 & -4 & -2 & 0 & 0 & 0 \\
-4 & 1 & -2 & -4 & 0 & 0 & 0 \\
-5 & 1 & -3 & -1 & 0 & 0 & 0
\end{array}\right)$$

The first column contains indices of variables, which are currently in the
basis. Negative indices mark slack variables and positive indices marks actual
variables (all actual variables are initially out of the basis). The second
column contains value of variables in the basis. The rest of the tableaux
contain coefficients of slack variables and actual variables.

Cells initially containing zeros represents coefficients of slack variables.
This corresponds to the fact that all slack variables are initially in the
basis. In more detail, the third and fourth column of first two rows contain
coefficients of slack variables $s_{1}$ and $s_{2}$, respectively. The fifth,
sixth and seventh column of the last three rows contain coefficients of slack
variables $s_{3}, s_{4}$ and $s_{5}$, respectively.

The used implementation layout of the tableaux is similar to the one presented
in \cite{Codenotti08}, but I put both matrices into a single tableaux (I do not
have to pass two matrices, since one is enough) and I use a different
indexing scheme.

\subsubsection{Computation}

Once the tableaux is initialized, we can begin with the computation. The used
method is iterative and is called \emph{pivoting}. At each step, one variable
\emph{will be brought into the basis} (pivot) and \emph{one variable will left
the basis} (it will tell us what the next pivot will be). If a variable $x_{1}$
is to enter the basis, we will search for an equation in which this variable
has the lowest \emph{min-ratio} (lowest negated value of the current basis
value divided by the entering variable coefficient). The right side of the
equation, in which the variable entered the basis, will be then substituted to
other equations.

The variable that left the basis (was ``replaced" with the variable that
entered the basis) will be used to compute the next variable that will enter
the basis. If variable $s_{i}$ just left the basis, then $x_{i}$ will be
brought in in the next step and vice versa.

As I use $x_{1}$ as an initial variable that enters the basis\footnote{The
index does not matter --- you could pick any $x_{i}$ \cite{Pritchard08}.}, we
will keep pivoting until $x_{1}$ or $s_{1}$ leaves the basis.

I will present the very first step of the computation on our game in both
representation (the algebraic one and the implemented one). Next steps
will be only shown from the implementation point of view.

As shown in the previous subsection, the following system is the tableaux
corresponding to our game (in the algebraic form).

$$\begin{array}{r r r r r r r r}
s_{1} = 1 &         &         & -3x_{3} & -5x_{4} & -6x_{5} \\
s_{2} = 1 &         &         & -6x_{3} & -x_{4}  & -5x_{5} \\
s_{3} = 1 & -4x_{1} & -2x_{2} &         &         &         \\
s_{4} = 1 & -2x_{1} & -4x_{2} &         &         &         \\
s_{5} = 1 & -3x_{1} & -x_{2}  &         &         &
\end{array}$$

Lets start pivoting and bring $x_{1}$ into the basis. The lowest ratio has
$x_{1}$ in the third equation ($-1/(-4) = 1/4$), so we have to pick
$x_{1}$ in this equation. We will move $x_{1}$ to the left side of the equation
and $s_{3}$ to the right side of the equation, properly divide all coefficients
and then we will substitute $x_{1}$ into the fourth and fifth equation, thus
obtaining the following system.

$$\begin{array}{r r r r r r r r}
s_{1} & = & 1 &             &           & -3x_{3} & -5x_{4} & -6x_{5} \\
s_{2} & = & 1 &             &           & -6x_{3} & -x_{4}  & -5x_{5} \\
x_{1} & = & 1/4 & -1/4s_{3} & -1/2x_{2} &         &         &         \\
s_{4} & = & 1/2 & 1/2s_{3}  & -3x_{2}   &         &         &         \\
s_{5} & = & 1/4 & 3/4s_{3}  & 1/2x_{2}  &         &         &
\end{array}$$

The situation in the implemented tableaux is analogical. We will brought
$x_{1}$ ($1$) into the basis and $s_{3}$ ($-3$) will leave the basis.

$$\left(\begin{array}{r r r r r r r}
-1 & 1 & 0 & 0 & -3 & -5 & -6 \\
-2 & 1 & 0 & 0 & -6 & -1 & -5 \\
1 & 1/4 & 0 & -1/2 & -1/4 & 0 & 0 \\
-4 & 1/2 & 0 & -3/1 & 1/2 & 0 & 0 \\
-5 & 1/4 & 0 & 1/2 & 3/4 & 0 & 0 \\
\end{array}\right)$$

Since $s_{3}$ ($-3$) left the basis, $x_{3}$ ($3$) will enter the basis in the
next step.

$$\left(\begin{array}{r r r r r r r}
-1 & 1/2 & 0 & 1/2 & 0 & -9/2 & -7/2 \\
3 & 1/6 & 0 & -1/6 & 0 & -1/6 & -5/6 \\
1 & 1/4 & 0 & -1/2 & -1/4 & 0 & 0 \\
-4 & 1/2 & 0 & -3/1 & 1/2 & 0 & 0 \\
-5 & 1/4 & 0 & 1/2 & 3/4 & 0 & 0 \\
\end{array}\right)$$

In the previous step, $s_{2}$ ($-2$) left the basis, so $x_{2}$ ($2$) will
enter the basis.

$$\left(\begin{array}{r r r r r r r}
-1 & 1/2 & 0 & 1/2 & 0 & -9/2 & -7/2 \\
3 & 1/6 & 0 & -1/6 & 0 & -1/6 & -5/6 \\
1 & 1/6 & 0 & 0 & -1/3 & 1/6 & 0 \\
2 & 1/6 & 0 & 0 & 1/6 & -1/3 & 0 \\
-5 & 1/3 & 0 & 0 & 5/6 & -1/6 & 0
\end{array}\right)$$

$s_{4}$ ($-4$) left the basis, so $x_{4}$ ($4$) will enter the basis.

$$\left(\begin{array}{r r r r r r r}
4 & 1/9 & -2/9 & 1/9 & 0 & 0 & -7/9 \\
3 & 4/27 & 1/27 & -5/27 & 0 & 0 & -19/27 \\
1 & 1/6 & 0 & 0 & -1/3 & 1/6 & 0 \\
2 & 1/6 & 0 & 0 & 1/6 & -1/3 & 0 \\
-5 & 1/3 & 0 & 0 & 5/6 & -1/6 & 0 \\
\end{array}\right)$$

As you can see, in the previous step, $s_{1}$ ($-1$) left the basis, which
means that we are done and the equilibrium was found.

\subsection{Obtaining the equilibrium}

Once we are done with pivoting, we can get the equilibrium from the resulting
tableaux. Strategy probabilities are in the second column, but we have to
properly interpret these values. If the value is out of the interval $\langle
0; 1 \rangle$, then the probability will be $0$ (probability cannot be negative
or higher that $1$). The values in the first column mark the position of that
probability. If the position is negative, then the probability on that position
in absolute value will be $0$.

Bearing that in mind, the equilibrium from our game is $((1/6, 1/6), (4/27,
1/9, 0))$. As you can see, the probabilities are somewhat weird, because their
sum is not equal to $1$. Therefore, we have to \emph{normalize} the equilibrium
using the same procedure as for vector normalization \cite{Pritchard08}.

$$normalized(x) = \left(\sum_{i = 1}^{|x|}x_{i}\right)^{-1}x$$

After the normalization, the equilibrium in mixed strategies in our game is
$((1/2, 1/2), (4/7, 3/7, 0))$.

\subsection{Output}

After the equilibrium is computed, both matrices and the found equilibrium is
printed to the standard output.

\subsubsection{Limitations}
\label{sec:Limitations}

The Lemke-Howson algorithm has some limitations. It finds only one mixed Nash
equilibrium per game and it works only for two-person games, which are
non-degenerative \cite{Stengel02, Pritchard08}.

\subsection{Tests}

Module tests were written using standard Python module for unit testing ---
\texttt{unittest}. Each module has an associated testing module. There is a
separate test case for each function or class.

\section{Testing and experiments}
\label{sec:Testing}

The implementation was thoroughly tested using unit-tests. There are 191 tests
in total, including overall program tests (computing mixed Nash equilibrium for
various games). All tests can automatically evaluate whether they were
successful or not, so they do not need any manual inspection (the testing
script prints just \texttt{OK} if everything went correctly, or an error
message describing the error). To run the tests, run \texttt{python
run-tests.py}. As an example, overall algorithm tests were written in the
following fashion.

\begin{verbatim}
    def scenarioValidRun(self, m1, m2, expEq):
        eq = lh.lemkeHowson(m1, m2)
        self.assertEqual(expEq, eq)

    def testEx1ValidRun(self):
        expEq = ((r.Rational(1, 2), r.Rational(1, 2)),
                 (r.Rational(1, 2), r. Rational(1, 2)))
        self.scenarioValidRun(EX1_M1, EX1_M2, expEq)

    def testEx2ValidRun(self):
        expEq = ((r.Rational(6, 13), r.Rational(3, 13), r.Rational(4, 13)),
                 (r.Rational(1, 9), r.Rational(3, 9), r.Rational(5, 9)))
        self.scenarioValidRun(EX2_M1, EX2_M2, expEq)

    ...
\end{verbatim}

There is a scenario which takes two matrices and an expected equilibrium. It
runs the algorithm and then compares the computed result with the expected
result. There are tests for several games, all of them written in this style.

If you want to try your own experiments, there are several example games
bundled in the archive (\texttt{game*.txt} files). You can freely modify them
to create new games (to create new games, just follow the format of the example
games). For more information about the usage of the program and description of
the syntax of the input, please see the \texttt{README} file.

\section{Conclusion}
\label{sec:Conclusion}

With proper supportive modules (matrices and rational numbers --- my own
implementation was used), the implementation of the Lemke-Howson turned out to
be fairly straightforward. My goal was to create a clear, working
implementation of this algorithm for finding mixed Nash equilibria in
two-person games. To achieve this, test driven development was used during the
development. Program modules were thoroughly tested using unit-tests, which
contain 191 tests in total. The program itself was successfully tested on
various systems, including Debian GNU/Linux and Microsoft Windows XP. Game
examples were bundled in the project archive.

\subsection*{Metrices}

\noindent
Source files: 12 (including tests) \\
Source lines of code: 1962 (without empty lines, including tests)

% References
\begin{thebibliography}{9}
	\bibitem{Lemke64} C. E. Lemke and J. J. T. Howson. \textit{Equilibrium
	points of bimatrix games}. SIAM Journal on Applied Mathematics,
	12(2):413-423, 1964.

	\bibitem{Stengel02} B. von Stengel. \textit{Computing equilibria for
	two-person games}. In R. J. Aumann and S. Hart, editors, Handbook of Game
	Theory, Vol. 3, pages 1723-1759, 2002.

	\bibitem{Pritchard08} D. Pritchard. \textit{The Lemke-Howson algorithm}.
	Notes on the Game Theory. [online]
	\url{http://www.math.uwaterloo.ca/~dagpritc/co456/lemke-howson.pdf}, 2008.

	\bibitem{Codenotti08} B. Codenotti, S. D. Rossi, M. Pagan. \textit{An
	experimental analysis of Lemke-Howson algorithm}. CoRR abs/0811.3247, 2008.
\end{thebibliography}

\end{document}
